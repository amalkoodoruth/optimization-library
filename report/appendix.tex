\clearpage
\section{Appendix : Code Listings}
\begin{lstlisting}
---------------------------Optimization Utils ----------------
import numpy as np

def get_step_size(p, x, s, gamma=1.5, mu=0.8):
    """
    Determines the step size using Armijo algorithm
    :param p: Problem
        The problem to be optimized
    :param x: numpy 2D array
        The point we are at
    :param s: numpy 2D array
        Search direction
    :param gamma: float
        Parameter for increasing step size
    :param mu: float
        Parameter for decreasing step size
    :return: float
        Step size
    """
    w = 1  # Default step size

    k_g = 0  # Power of gamma
    k_m = 0  # Power of mu

    # Precompute cost and gradient to save time
    vx = p.get_cost(x)
    gx_s = p.get_grad(x) @ s

    def v_bar(w):
        return vx + 0.5 * w * gx_s

    while p.get_cost(x + gamma ** k_g * s) < v_bar(gamma ** k_g):
        k_g += 1
        w = gamma ** k_g

    while p.get_cost(x + mu ** k_m * gamma ** k_g * s) > v_bar(mu ** k_m * gamma ** k_g):
        k_m += 1
        w = mu ** k_m * gamma ** k_g

    return w

def check_cone_condition(p, x, s, theta=89):
    """
    Checks the cone condition at a point
    :param p: Problem
        The problem to be optimized
    :param x: numpy 2D array
        The point we are at
    :param s: numpy 2D array
        Search direction
    :param theta: float
        Acceptable angle with gradient in degrees
    :return: bool
        True if s within theta degrees of gradient at x
    """
    gx = p.get_grad(x)  # get gradient at x
    cos_phi = (-gx @ s) / (np.linalg.norm(s) * np.linalg.norm(gx))
    cos_theta = np.cos(theta * 2 * np.pi / 360)

    return (cos_phi > cos_theta)

def get_cost_norm(p, x):
    """
    Evaluates the cost at d
    :param p: Problem
        The problem to be optimized
    :param x: numpy 2D array
        The point at which we want to get the cost
    :return: float
        The cost at x
    """
    cost = 0
    if p.get_eq_const() is not None:
        cost = cost + np.linalg.norm(p.get_eq_const(x))**2
    if p.get_ineq_const() is not None:
        ineq_x = p.get_ineq_const(x)
        c = np.minimum(np.zeros(np.shape(ineq_x)), ineq_x)
        cost = cost + np.linalg.norm(c)**2
    return np.sqrt(cost)

def get_fd_hessian(f, x, h=1e-8):
    """
    The finite difference approximation of the Hessian at x
    :param f: function of x
        Function whose Hessian we want to evaluate
    :param x: numpy 2D array
        Point at whihc we want to evaluate the Hessian
    :param h: float
        Step size
    :return: 2D numpy matrix
        Hessian at x
    """
    dim = np.max(np.shape(x))
    I = np.eye(dim)
    H = np.zeros((dim, dim))

    for i in range(0, dim):
        for j in range(0, dim):
            H[i, j] = (f(x + h * I[:, [i]] + h * I[:, [j]]) \
                       - f(x + h * I[:, [i]]) - f(x + h * I[:, [j]]) \
                       + f(x)) / h ** 2

    return 0.5 * (H + H.T)

def get_fd_grad(f, x, h=1e-8):
    """
    Finite difference approximation of gradient at x
    :param f: function of x
        function whose gradient we want to evaluate
    :param x: numpy 2D array
        point at which we want to approximate the gradient
    :param h: float
        step size
    :return: numpy 2D array
        gradient at x 
    """
    dim = np.max(np.shape(x))
    grad_gen = ((f(x + h * np.eye(dim)[:, [i]]) - f(x)) / h
                for i in range(0, dim))
    grad = np.expand_dims(np.fromiter(grad_gen, np.float64), axis=0)
    return grad

--------------------------------------Optimization------------------------------------------
import numpy as np
from functools import partial
from optimization_utils import get_step_size,check_cone_condition,get_cost_norm,get_fd_hessian,get_fd_grad

class Problem:
    """
    This class defines an optimization problem
    """
    def __init__(self, cost, grad='fd', grad_step=1e-8,
                 eq_const=None, ineq_const=None):
        """
        Constructor for the otpimization problem

        :param cost: function with 2D NumPy array as input
            Objective function of the problem
        :param grad: function with 2D NumPy array as input
            Gradient function of the problem.
            Default = 'fd', finite difference used
        :param grad_step: float
            Step size for finite difference gradient.
            Only required if :param grad= 'fd'.
            Default = 1e-8
        :param eq_const: list of functions
            Equality constraints of the problem
        :param ineq_const: list of functions
            Inequality constraints of the problem
        """
        self._cost = cost

        if grad == 'fd':
            # finite difference method specified
            self._grad_step = grad_step  # default 1e-8
            self._grad = partial(get_fd_grad, self.get_cost, h=self._grad_step)

        else:
            # gradient specified, don't need step size
            self._grad_step = None
            self._grad = grad

        self._eq_const = eq_const
        self._ineq_const = ineq_const

    def get_cost(self, x=None):
        """
        Returns the cost of the problem if x is specified.
        Else, returns the cost function.

        :param x: 2D numpy array
            Point at which the cost is calculated.
            Default = None
        :return: function or float
            If x is None, returns the cost function.
            Else, returns the cost at x.
        """
        if x is not None:
            return self._cost(x)
        else:
            return self._cost

    def get_grad(self, x=None):
        """
        Returns the gradient of the problem if x is specified.
        Else, returns the gradient function.

        :param x: 2D numpy array
            Point at which the gradient is calculated.
            Default = None
        :return: function or float
            If x is None, returns the gradient function.
            Else, returns the gradient at x.
        """
        if x is not None:
            return self._grad(x)
        else:
            return self._grad

    def get_eq_const(self, x=None):
        """
        Equality constraints of the problem.
        If x is None, returns an array of functions.
        Else, returns an array of equality constraints evaluated at x

        :param x: 2D numpy array
            Point at which the equality constraints are evaluated.
        :return: 2D numpy array
            Array of functions if x = None.
            Else, array of equality constraints evaluated at x
        """
        if self._eq_const is not None:
            if x is not None:
                return np.array([[eq(x)] for eq in self._eq_const])
            else:
                return np.array([eq for eq in self._eq_const])
        else:
            return None

    def get_ineq_const(self, x=None):
        """
        Inequality constraints of the problem.
        If x is None, returns an array of functions.
        Else, returns an array of inequality constraints evaluated at x

        :param x: 2D numpy array
            Point at which the inequality constraints are evaluated.
        :return: 2D numpy array
            Array of functions if x = None.
            Else, array of inequality constraints evaluated at x
        """
        if self._ineq_const is not None:
            if x is not None:
                return np.array([[ineq(x)] for ineq in self._ineq_const])
            else:
                return np.array([ineq for ineq in self._ineq_const])
        else:
            return None

    def num_eq_const(self):
        """
        The number of equality constraints.

        :return: int
            The number of equality constraints.
        """
        if self._eq_const is not None:
            return np.max(np.shape(self._eq_const))
        else:
            return 0

    def num_ineq_const(self):
        """
        The number of inequality constraints.

        :return: int
            The number of inequality constraints.
        """
        if self._ineq_const is not None:
            return np.max(np.shape(self._ineq_const))
        else:
            return 0

def steepest_descent(p, x, tol=1e-6, max_iter=999, hist=False):
    """
    Steepest Descent algorithm

    :param p: Problem
        Optimization problem to minimize
    :param x: 2D numpy array
        Initial guess
    :param tol: float
        Tolerance of the algorithm. If norm of gradient is less than tol,
        iterations stop.
        Default = 1e-6
    :param max_iter: int
        Maximum number of iterations before stopping
        Default = 999
    :param hist: bool
        Flag to return history of x or not
        Default = False
    :return: 2D or 3D numpy array
        If hist = False, returns the last value of x.
        Else, returns an array with values of x after each iteration
    """
    i = 0
    x_hist = []
    while np.linalg.norm(p.get_grad(x)) > tol:
        if i > max_iter:
            break
        s = -p.get_grad(x).T # get the gradient at x
        w = get_step_size(p, x, s) # get step size using Armijo algorithm
        x_hist.append(x)
        x = x + w * s # gradient descent
        i += 1 # increment iteration

    return x if not hist else np.array(x_hist)

def conjugate_gradient(p, x, tol=1e-6, rst_iter=99, max_iter=999, hist=False):
    """
    Conjugate gradient algorithm

    :param p: Problem
        Optimization problem to minimize
    :param x: 2D numpy array
        Initial guess
    :param tol: float
        Tolerance of the algorithm. If norm of gradient is less than tol,
        iterations stop.
        Default = 1e-6
    :param rst_iter: int
        Number of iterations before resetting search direction
        and restarting
        Default = 99
    :param max_iter: int
        Maximum total number of iterations
        Default = 999
    :param hist: bool
        Flag to return history of x or not
        Default = False
    :return: 2D or 3D numpy array
        If hist = False, returns the last value of x.
        Else, returns an array with values of x after each iteration
    """
    i = 0
    x_hist = []
    s = -p.get_grad(x).T
    while np.linalg.norm(p.get_grad(x)) > tol:
        if i > rst_iter or not check_cone_condition(p, x, s):
            i = 0
            s = -p.get_grad(x).T # get gradient at x
        # elif not check_cone_condition(p, x, s):
        #     i = 0
        #     s = -p.grad(x).T # get gradient at x
        elif i > max_iter:
            break
        w = get_step_size(p, x, s) # get step size using Armijo algorithm
        x_prv = x
        x_hist.append(x)
        x = x_prv + w * s
        beta = ((p.get_grad(x) - p.get_grad(x_prv)) @ p.get_grad(x).T) \
            / (p.get_grad(x_prv) @ p.get_grad(x_prv).T)
        s = -p.get_grad(x).T + beta * s
        i += 1

    return x if not hist else np.array(x_hist)

def secant(p, x, tol=1e-6, H=None, rst_iter=99, max_iter=999, hist=False):
    """
    Secant optimization algorithm

    :param p: Problem
        Optimization problem to minimize
    :param x: 2D numpy array
        Initial guess
    :param tol: float
        Tolerance of the algorithm. If norm of gradient is less than tol,
        iterations stop.
        Default = 1e-6
    :param H: 2D numpy matrix
        Initial guess at Hessian. If None, set to identity
        Default = None
    :param rst_iter: int
        Number of iterations before resetting search direction
        and restarting
        Default = 99
    :param max_iter: int
        Maximum total number of iterations
        Default = 999
    :param hist: bool
        Flag to return history of x or not
        Default = False
    :return: 2D or 3D numpy array
        If hist = False, returns the last value of x.
        Else, returns an array with values of x after each iteration
    """
    if H is None:
        H = np.eye(np.max(np.shape(x)))

    i = 0
    x_hist = []
    while np.linalg.norm(p.get_grad(x)) > tol:
        s = -H @ p.get_grad(x).T
        if i > rst_iter or not check_cone_condition(p, x, s):
            i = 0
            s = -p.get_grad(x).T # get gradient at x
        # elif not check_cone_condition(p, x, s):
        #     i = 0
        #     s = -p.grad(x).T
        elif i > max_iter:
            break
        w = get_step_size(p, x, s)
        x_prv = x
        x_hist.append(x)
        x = x_prv + w * s
        # Davidon-Fletcher-Powell (DFP) Algorithm
        dx = x - x_prv
        dg = p.get_grad(x) - p.get_grad(x_prv)
        H = H + (dx @ dx.T) / (dx.T @ dg.T) \
            - ((H @ dg.T) @ (H @ dg.T).T) / (dg @ H @ dg.T)
        i += 1

    return x if not hist else np.array(x_hist)

def penalty_function(p, x0, tol=1e-6, tol_cost=1e-4, sigma_max=1e6, hist=False):
    """
    Constrained optimization algorithm using penalty functions
    :param p: Problem
        Constrained optimization problem to minimize
    :param x0: 2D numpy array
        Initial guess
    :param tol: float
        Tolerance of the algorithm. If norm of gradient is less than tol,
        iterations stop.
        Default = 1e-6
    :param tol_cost: float
        Tolerance of cost. If norm of cost is less than tol_cost,
        iteration stops.
        Default = 1e-4
    :param sigma_max: float
        Maximum value of sigma before iteration stops
        Default = 1e6
    :param hist: bool
        Flag to return history of x or not
        Default = False
    :return: 2D or 3D numpy array
        If hist = False, returns the last value of x.
        Else, returns an array with values of x after each iteration
    """

    def phi(p, sigma, x):
        cost = p.get_cost(x) # get cost at x
        if p.get_eq_const() is not None:
            cost = cost + 0.5 * sigma * np.linalg.norm(p.get_eq_const(x))**2
        if p.get_ineq_const() is not None:
            ineq_x = p.get_ineq_const(x)
            c = np.minimum(np.zeros(np.shape(ineq_x)), ineq_x)
            cost = cost + 0.5 * sigma * np.linalg.norm(c)**2
        return cost

    # def cost_norm(x):
    #     cost = 0
    #     if p.eq_const() is not None:
    #         cost = cost + np.linalg.norm(p.eq_const(x))**2
    #     if p.ineq_const() is not None:
    #         ineq_x = p.ineq_const(x)
    #         c = np.minimum(np.zeros(np.shape(ineq_x)), ineq_x)
    #         cost = cost + np.linalg.norm(c)**2
    #     return np.sqrt(cost)

    sigma = 1
    x = x0 # initialize first guess
    x_hist = []

    while get_cost_norm(p, x) > tol_cost:
        up = Problem(partial(phi, p, sigma))
        x_hist.append(x)
        x = steepest_descent(up, x0, tol=tol)
        if sigma >= sigma_max:
            break
        sigma *= 10

    return x if not hist else np.array(x_hist)

def barrier_function(p, x0, tol=1e-6, tol_const=1e-4, sigma_max=1e6,
                     r_min=1e-6, mode='inv', hist=False):
    """
    Constrained optimization algorithm using barrier functions
    :param p: Problem
        Constrained optimization problem to minimize
    :param x0: 2D numpy array
        Initial guess
    :param tol: float
        Tolerance of the algorithm. If norm of gradient is less than tol,
        iterations stop.
        Default = 1e-6
    :param tol_cost: float
        Tolerance of cost. If norm of cost is less than tol_cost,
        iteration stops.
        Default = 1e-4
    :param sigma_max: float
        Maximum value of sigma before iteration stops
        Default = 1e6
    :param r_min: float
        Minimum value of r before iteration stops
        Default = 1e-6
    :param mode: String
        Mode of barrier function. 'inv' for inverse barrier function
        or 'log' for logarithmic barrier function
    :param hist: bool
        Flag to return history of x or not
        Default = False
    :return: 2D or 3D numpy array
        If hist = False, returns the last value of x.
        Else, returns an array with values of x after each iteration
    """
    def phi(p, sigma, r, x):
        cost = p.get_cost(x)
        if p.get_eq_const() is not None:
            cost = cost + 0.5 * sigma * np.linalg.norm(p.get_eq_const(x))**2
        if p.get_ineq_const() is not None:
            ineq_x = p.get_ineq_const(x)
            if mode == 'log':
                cost = cost - r * np.sum(np.log(ineq_x))
            else:
                cost = cost + r * np.sum(np.reciprocal(ineq_x))
        return cost

    # def cost_norm(x):
    #     cost = 0
    #     if p.eq_const() is not None:
    #         cost = cost + np.linalg.norm(p.eq_const(x))**2
    #     if p.ineq_const() is not None:
    #         ineq_x = p.ineq_const(x)
    #         c = np.minimum(np.zeros(np.shape(ineq_x)), ineq_x)
    #         cost = cost + np.linalg.norm(c)**2
    #     return np.sqrt(cost)

    sigma = 1
    r = 1
    x = x0
    x_hist = []

    while get_cost_norm(p, x) > tol_const:
        up = Problem(partial(phi, p, sigma, r))
        x_hist.append(x)
        x = steepest_descent(up, x0, tol=tol)
        if sigma >= sigma_max or r <= r_min:
            break
        sigma *= 10
        r *= 0.1

    return x if not hist else np.array(x_hist)

def augmented_lagrange(p, x0, tol=1e-6, tol_cost=1e-6, sigma_max=1e12, hist=False):
    """
    Constrained optimization algorithm using augmented Lagrange method
    :param p: Problem
        Constrained optimization problem to minimize
    :param x0: 2D numpy array
        Initial guess
    :param tol: float
        Tolerance of the algorithm. If norm of gradient is less than tol,
        iterations stop.
        Default = 1e-6
    :param tol_cost: float
        Tolerance of cost. If norm of cost is less than tol_cost,
        iteration stops.
        Default = 1e-6
    :param sigma_max: float
        Maximum value of sigma before iteration stops
        Default = 1e6
    :param hist: bool
        Flag to return history of x or not
        Default = False
    :return: 2D or 3D numpy array
        If hist = False, returns the last value of x.
        Else, returns an array with values of x after each iteration
    """
    def phi(p, lmb, sgm, x):
        cost = p.get_cost(x)

        n_e = p.num_eq_const()
        n_i = p.num_ineq_const()
        n_c = n_e + n_i

        lmb_e = lmb[0:n_e, :]
        lmb_i = lmb[n_e:n_c, :]
        sgm_e = sgm[0:n_e, :]
        sgm_i = sgm[n_e:n_c, :]

        if p.get_eq_const() is not None:
            c_e = p.get_eq_const(x)
            cost = cost - sum(lmb_e * c_e) + 0.5 * sum(sgm_e * c_e**2)

        if p.get_ineq_const() is not None:
            c_i = p.get_ineq_const(x)
            p_i = np.array([-lmb_i[i] * c_i[i] + 0.5 * sgm_i[i] * c_i[i]**2 \
                            if c_i[i] <= lmb_i[i] / sgm_i[i] \
                            else -0.5 * lmb_i[i]**2 / sgm_i[i] \
                            for i in range(0, n_i)])
            cost = cost + sum(p_i)

        return cost

    x_hist = []

    n_e = p.num_eq_const()
    n_i = p.num_ineq_const()
    n_c = n_e + n_i

    lmb = np.zeros((n_c, 1))
    sgm = np.ones((n_c, 1))

    x = x0
    c = 1e12 * np.ones((n_c, 1))

    while np.linalg.norm(c) > tol_cost:
        # Create new problem to solve, but unconstrained
        up = Problem(partial(phi, p, lmb, sgm))
        x_hist.append(x)
        x = steepest_descent(up, x0, tol=tol)

        # Concatenate costs
        c_prv = c
        c_e = p.get_eq_const(x)
        c_i = p.get_ineq_const(x)
        if c_e is not None and c_i is not None:
            c = np.concatenate((c_e, c_i), axis=0)
        elif c_e is not None:
            c = c_e
        elif c_i is not None:
            c = c_i

        # Make sure sigma is not too big
        if any(sgm >= sigma_max):
            break

        # Update sigma
        if np.linalg.norm(c, np.inf) > 0.25 * np.linalg.norm(c_prv, np.inf):
            for i in range(0, n_c):
                if np.abs(c[i]) > 0.25 * np.linalg.norm(c_prv, np.inf):
                    sgm[i] *= 10
            continue

        lmb = lmb - (sgm * c)

    return x if not hist else np.array(x_hist)

def lagrange_newton(p, x0, tol=1e-6, hist=False):
    """
    Constrained optimization algorithm using augmented Lagrange-Newton method
    :param p: Problem
        Constrained optimization problem to minimize
    :param x0: 2D numpy array
        Initial guess
    :param tol: float
        Tolerance of the algorithm. If norm of gradient is less than tol,
        iterations stop.
        Default = 1e-6
    :param hist: bool
        Flag to return history of x or not
        Default = False
    :return: 2D or 3D numpy array
        If hist = False, returns the last value of x.
        Else, returns an array with values of x after each iteration
    """
    x_hist = []

    n_e = p.num_eq_const()
    n_i = p.num_ineq_const()
    n_c = n_e + n_i

    def W(x, lmb):
        lmb_e = lmb[0:n_e, :]
        lmb_i = lmb[n_e:n_c, :]
        hess_f = get_fd_hessian(p.get_cost, x)
        hess_c_e = - np.sum([lmb_e[i] * get_fd_hessian(p.get_eq_const()[i], x)
            for i in range(0, n_e)])
        hess_c_i = - np.sum([lmb_i[i] * get_fd_hessian(p.get_ineq_const()[i], x)
            for i in range(0, n_i)])
        hess = hess_f + hess_c_e + hess_c_i
        return hess

    def A(x):
        grad_e = np.array([np.squeeze(get_fd_grad(p.get_eq_const()[i], x))
                for i in range (0, n_e)])
        grad_i = np.array([np.squeeze(get_fd_grad(p.get_ineq_const()[i], x))
                for i in range (0, n_i)])
        if n_e != 0 and n_i != 0:
            grad = np.concatenate((grad_e, grad_i), axis=0)
        elif n_e != 0:
            grad = grad_e
        elif n_i != 0:
            grad = grad_i
        return grad

    x = x0
    lmb = np.zeros((n_c, 1))

    # Concatenate costs
    c_e = p.get_eq_const(x)
    c_i = p.get_ineq_const(x)
    if c_e is not None and c_i is not None:
        c = np.concatenate((c_e, c_i), axis=0)
    elif c_e is not None:
        c = c_e
    elif c_i is not None:
        c = c_i

    delta_x = 1e12

    while delta_x  > tol:

        # Compute KKT matrix
        KKT = np.block([
            [W(x, lmb), -A(x).T],
            [-A(x), np.zeros((n_c, n_c))]
        ])

        # Compute gradient augmented with constraints
        if n_e != 0 and n_i != 0:
            f = np.block([
                [-get_fd_grad(p.get_cost, x).T + A(x).T @ lmb],
                [p.get_eq_const(x)],
                [p.get_ineq_const(x)]
            ])
        elif n_e != 0:
            f = np.block([
                [-get_fd_grad(p.get_cost, x).T + A(x).T @ lmb],
                [p.get_eq_const(x)]
            ])
        elif n_i != 0:
            f = np.block([
                [-get_fd_grad(p.get_cost, x).T + A(x).T @ lmb],
                [p.get_ineq_const(x)]
            ])

        x_prv = x
        # Invert KKT matrix to get x and lambda increments
        X = np.linalg.solve(KKT, f)
        dim = np.max(np.shape(x))
        x_hist.append(x)
        # Apply x and lambda increments
        x = x + X[:dim, :]
        lmb = lmb + X[dim:, :]

        c_e = p.get_eq_const(x)
        c_i = p.get_ineq_const(x)

        if c_e is not None and c_i is not None:
            c = np.concatenate((c_e, c_i), axis=0)
        elif c_e is not None:
            c = c_e
        elif c_i is not None:
            c = c_i

        # Check distance from previous x
        delta_x = np.linalg.norm(x - x_prv)

    return x if not hist else np.array(x_hist)
    
------------------------------Testing Algorithm------------------------------------------------------------
import unittest
import time
import numpy as np
import matplotlib.pyplot as plt
import optimization

np.set_printoptions(precision=20, linewidth=120)

class TestProblemAGrad(unittest.TestCase):
    def setUp(self):
        # Set up problem
        a = 5
        b = np.array([[1], [4], [5], [4], [2], [1]])
        C = 2 * np.array([[9, 1, 7, 5, 4, 7],
                          [1, 11, 4, 2, 7, 5],
                          [7, 4, 13, 5, 0, 7],
                          [5, 2, 5, 17, 1, 9],
                          [4, 7, 0, 1, 21, 15],
                          [7, 5, 7, 9, 15, 27]])

        v = lambda x: a + b.T @ x + 0.5 * x.T @ C @ x
        del_v = lambda x: b.T + x.T @ C

        # Creating instance of problem
        self.p = optimization.Problem(v, del_v)
        # Store known solution since this is a quadratic. Not used here
        self.x_opt = -np.linalg.solve(C, b)

    def test_steepest_descent(self):
        # Initial guess
        x = np.array([[0], [0], [0], [0], [0], [0]])

        # Start timer
        start = time.time()

        # Steepest descent algorithm
        x_values = optimization.steepest_descent(self.p, x, hist=True)
        end = time.time()

        # Evaluate gradient at each iteration
        g = np.array([np.linalg.norm(self.p.get_grad(x_values[i]))
                      for i in range(len(x_values))])

        # Evaluate cost at each point

        # Plotting and saving
        fig = plt.figure()
        plt.plot(np.arange(len(x_values)), g)
        plt.xlabel('Iteration')
        plt.ylabel('Norm of Gradient')
        fig.savefig('./fig/sd-pA-grad.eps', format='eps')
        # Print out stats
        print('\nProblem A, Steepest Descent (Exact Gradient)')
        print('arg min v(x) =\n', x_values[-1])
        print('time =\n', end - start, 's')

        print(f'x shape: {x_values.shape}')




    def test_conjugate_gradient(self):
        # initial guess
        x = np.array([[0], [0], [0], [0], [0], [0]])

        # Start timer
        start = time.time()

        # Conjugate gradient algorithm
        x_values = optimization.conjugate_gradient(self.p, x, hist=True)
        end = time.time()

        # Evaluate gradient at each iteration
        g = np.array([np.linalg.norm(self.p.get_grad(x_values[i]))
                      for i in range(len(x_values))])

        # Plotting and saving
        fig = plt.figure()
        plt.plot(np.arange(len(x_values)), g)
        plt.xlabel('Iteration')
        plt.ylabel('Norm of Gradient')
        fig.savefig('./fig/cg-pA-grad.eps', format='eps')
        print('\nProblem A, Conjugate Gradient (Exact Gradient)')
        print('arg min v(x) =\n', x_values[-1])
        print('time =\n', end - start, 's')

    def test_secant(self):
        # initial guess
        x = np.array([[0], [0], [0], [0], [0], [0]])

        # Start timer
        start = time.time()

        # Secant algorithm
        x_values = optimization.secant(self.p, x, hist=True)
        end = time.time()

        # Evaluate gradient at each iteration
        g = np.array([np.linalg.norm(self.p.get_grad(x_values[i]))
                      for i in range(len(x_values))])

        # Plotting and saving
        fig = plt.figure()
        plt.plot(np.arange(len(x_values)), g)
        plt.xlabel('Iteration')
        plt.ylabel('Norm of Gradient')
        fig.savefig('./fig/sec-pA-grad.eps', format='eps')
        print('\nProblem A, Secant (Exact Gradient)')
        print('arg min v(x) =\n', x_values[-1])
        print('time =\n', end - start, 's')

class TestProblemA(unittest.TestCase):

    def setUp(self):
        a = 5
        b = np.array([[1], [4], [5], [4], [2], [1]])
        C = 2 * np.array([[9,  1,  7,  5,  4,  7],
                          [1, 11,  4,  2,  7,  5],
                          [7,  4, 13,  5,  0,  7],
                          [5,  2,  5, 17,  1,  9],
                          [4,  7,  0,  1, 21, 15],
                          [7,  5,  7,  9, 15, 27]])

        v = lambda x : a + b.T @ x + 0.5 * x.T @ C @ x
        self.p = optimization.Problem(v)
        self.x_values = -np.linalg.solve(C, b)

    def test_sd(self):
        x = np.array([[0], [0], [0], [0], [0], [0]])
        start = time.time()
        x_values = optimization.steepest_descent(self.p, x, tol=1e-6, hist=True)
        end = time.time()
        g = np.array([np.linalg.norm(self.p.get_grad(x_values[i]))
            for i in range(len(x_values))])
        fig = plt.figure()
        plt.plot(np.arange(len(x_values)), g)
        plt.xlabel('Iteration')
        plt.ylabel('Norm of Gradient')
        plt.show()
        #fig.savefig('./fig/sd-pA.eps', format='eps')
        print('\nProblem A, Steepest Descent')
        print('arg min v(x) =\n', x_values[-1])
        print('time =\n', end - start, 's')

    def test_cg(self):
        x = np.array([[0], [0], [0], [0], [0], [0]])
        start = time.time()
        x_values = optimization.conjugate_gradient(self.p, x, tol=1e-6, hist=True)
        end = time.time()
        g = np.array([np.linalg.norm(self.p.get_grad(x_values[i]))
            for i in range(len(x_values))])
        fig = plt.figure()
        plt.plot(np.arange(len(x_values)), g)
        plt.xlabel('Iteration')
        plt.ylabel('Norm of Gradient')
        plt.show()
        fig.savefig('./fig/cg-pA.eps', format='eps')
        print('\nProblem A, Conjugate Gradient')
        print('arg min v(x) =\n', x_values[-1])
        print('time =\n', end - start, 's')

    def test_sec(self):
        x = np.array([[0], [0], [0], [0], [0], [0]])
        start = time.time()
        x_values = optimization.secant(self.p, x, tol=1e-6, hist=True)
        end = time.time()
        g = np.array([np.linalg.norm(self.p.get_grad(x_values[i]))
            for i in range(len(x_values))])
        fig = plt.figure()
        plt.plot(np.arange(len(x_values)), g)
        plt.xlabel('Iteration')
        plt.ylabel('Norm of Gradient')
        fig.savefig('./fig/sec-pA.eps', format='eps')
        print('\nProblem A, Secant')
        print('arg min v(x) =\n', x_values[-1])
        print('time =\n', end - start, 's')


class TestProblemB(unittest.TestCase):

    def setUp(self):
        v = lambda x : -np.sqrt((x[0, 0]**2 + 1) * (2 * x[1, 0]**2 + 1)) \
                       / (x[0, 0]**2 + x[1, 0]**2 + 0.5)
        self.x_opt = np.array([[0], [0]])
        self.p = optimization.Problem(v)


    def test_sd(self):
        x = np.array([[-1], [1]])
        start = time.time()
        x_values = optimization.steepest_descent(self.p, x, tol=1e-8, hist=True)
        end = time.time()
        g = np.array([np.linalg.norm(self.p.get_grad(x_values[i]))
            for i in range(len(x_values))])
        fig = plt.figure()
        plt.plot(np.arange(len(x_values)), g)
        plt.xlabel('Iteration')
        plt.ylabel('Norm of Gradient')
        fig.savefig('./fig/sd-pB.eps', format='eps')
        print('\nProblem B, Steepest Descent')
        print('arg min v(x) =\n', x_values[-1])
        print('time =\n', end - start, 's')

    def test_cg(self):
        x = np.array([[-2], [1]])
        start = time.time()
        x_values = optimization.conjugate_gradient(self.p, x, tol=1e-8, hist=True)
        end = time.time()
        g = np.array([np.linalg.norm(self.p.get_grad(x_values[i]))
            for i in range(len(x_values))])
        fig = plt.figure()
        plt.plot(np.arange(len(x_values)), g)
        plt.xlabel('Iteration')
        plt.ylabel('Norm of Gradient')
        fig.savefig('./fig/cg-pB.eps', format='eps')
        print('\nProblem B, Conjugate Gradient')
        print('arg min v(x) =\n', x_values[-1])
        print('time =\n', end - start, 's')

    def test_sec(self):
        x = np.array([[1], [1]])
        start = time.time()
        x_values = optimization.secant(self.p, x, tol=1e-8, hist=True)
        end = time.time()
        g = np.array([np.linalg.norm(self.p.get_grad(x_values[i]))
            for i in range(len(x_values))])
        fig = plt.figure()
        plt.plot(np.arange(len(x_values)), g)
        plt.xlabel('Iteration')
        plt.ylabel('Norm of Gradient')
        fig.savefig('./fig/sec-pB.eps', format='eps')
        print('\nProblem B, Secant')
        print('arg min v(x) =\n', x_values[-1])
        print('time =\n', end - start, 's')

class TestProblemC(unittest.TestCase):

    def setUp(self):
        a = 1
        b = np.array([[1], [2]])
        C = np.array([[12, 3], [3, 10]])
        v = lambda x : a + b.T @ x + x.T @ C @ x \
                       + 10 * np.log(1 + x[0, 0]**4) * np.sin(100 * x[0, 0]) \
                       + 10 * np.log(1 + x[1, 0]**4) * np.cos(100 * x[1, 0])
        self.p = optimization.Problem(v)

    def test_sd(self):
        x = np.array([[0], [0]])
        start = time.time()
        x_opt = optimization.steepest_descent(self.p, x, tol=1e-4, hist=True)
        end = time.time()
        g = np.array([np.linalg.norm(self.p.get_grad(x_opt[i]))
            for i in range(len(x_opt))])
        fig = plt.figure()
        plt.plot(np.arange(len(x_opt)), g)
        plt.xlabel('Iteration')
        plt.ylabel('Norm of Gradient')
        fig.savefig('./fig/sd-pC.eps', format='eps')
        print('\nProblem C, Steepest Descent')
        print('arg min v(x) =\n', x_opt[-1])
        print('time =\n', end - start, 's')

    def test_cg(self):
        x = np.array([[0], [0]])
        start = time.time()
        x_values = optimization.conjugate_gradient(self.p, x, tol=1e-4, hist=True)
        end = time.time()
        g = np.array([np.linalg.norm(self.p.get_grad(x_values[i]))
            for i in range(len(x_values))])
        fig = plt.figure()
        plt.plot(np.arange(len(x_values)), g)
        plt.xlabel('Iteration')
        plt.ylabel('Norm of Gradient')
        fig.savefig('./fig/cg-pC.eps', format='eps')
        print('\nProblem C, Conjugate Gradient')
        print('arg min v(x) =\n', x_values[-1])
        print('time =\n', end - start, 's')

    def test_sec(self):
        x = np.array([[0], [0]])
        start = time.time()
        x_values = optimization.secant(self.p, x, tol=1e-4, hist=True)
        end = time.time()
        g = np.array([np.linalg.norm(self.p.get_grad(x_values[i]))
            for i in range(len(x_values))])
        fig = plt.figure()
        plt.plot(np.arange(len(x_values)), g)
        plt.xlabel('Iteration')
        plt.ylabel('Norm of Gradient')
        fig.savefig('./fig/sec-pC.eps', format='eps')
        print('\nProblem C, Secant')
        print('arg min v(x) =\n', x_values[-1])
        print('time =\n', end - start, 's')

class TestProblemD(unittest.TestCase):

    def setUp(self):
        # Objective function
        v = lambda x: np.abs(x[0, 0] - 2) + np.abs(x[1, 0] - 2)
        # Constraints
        h1 = lambda x: x[0, 0] - x[1, 0]**2
        h2 = lambda x: x[0, 0]**2 + x[1, 0]**2 - 1
        # Create Problem with equality and inequality constraints in lists
        self.p = optimization.Problem(v, eq_const=[h2], ineq_const=[h1])
        # Known solution. Not used here
        self.x_opt = np.array([[np.sqrt(2)/2], [np.sqrt(2)/2]])

    def test_penalty_function(self, tol=1e-3, tol_const=1e-3):
        x0 = np.array([[0], [0]])
        start = time.time()
        x_values = optimization.penalty_function(self.p, x0, hist=True)
        end = time.time()
        g = np.array([np.linalg.norm(self.p.get_grad(x_values[i]))
            for i in range(len(x_values))])
        c_e = np.array([np.linalg.norm(self.p.get_eq_const(x_values[i]))
            for i in range(len(x_values))])
        c_i = np.array([np.linalg.norm(np.minimum(self.p.get_ineq_const(x_values[i]), 0))
            for i in range(len(x_values))])
        fig = plt.figure()
        plt.plot(np.arange(len(x_values)), g, label='Gradient Norm')
        #plt.plot(np.arange(len(x_values)), c_e, label='Equality Constraint Norm')
        #plt.plot(np.arange(len(x_values)), c_i, label='Inequality Constraint Norm')
        plt.xticks(np.arange(len(x_values)))
        plt.xlabel('Iteration')
        plt.legend()
        plt.show()
        fig.savefig('./fig/pe-pD.eps', format='eps')
        print('\nProblem D, Penalty Function')
        print('arg min v(x) =\n', x_values[-1])
        print('time =\n', end - start, 's')

    def test_barrier_function(self, tol=1e-3, tol_const=1e-3):
        x0 = np.array([[1], [1]])
        start = time.time()
        x_values = optimization.barrier_function(self.p, x0,mode = 'inv')
        end = time.time()
        g = np.array([np.linalg.norm(self.p.get_grad(x_values[i]))
            for i in range(len(x_values))])
        c_e = np.array([np.linalg.norm(self.p.get_eq_const(x_values[i]))
            for i in range(len(x_values))])
        c_i = np.array([np.linalg.norm(np.minimum(self.p.get_ineq_const(x_values[i]), 0))
            for i in range(len(x_values))])
        fig = plt.figure()
        plt.plot(np.arange(len(x_values)), g, label='Gradient Norm')
        #plt.plot(np.arange(len(x_values)), c_e, label='Equality Constraint Norm')
        #plt.plot(np.arange(len(x_values)), c_i, label='Inequality Constraint Norm')
        plt.xticks(np.arange(len(x_values)))
        plt.xlabel('Iteration')
        plt.legend()
        plt.show()
        fig.savefig('./fig/pe-bf.eps', format='eps')
        print('\nProblem D, Barrier Function')
        print('arg min v(x) =\n', x_values[-1])
        print('time =\n', end - start, 's')
    
    def test_aug_lag(self):
        x0 = np.array([[0], [1]])
        start = time.time()
        x_values = optimization.augmented_lagrange(self.p, x0, tol=1e-4, tol_cost=1e-4, hist=True)
        end = time.time()
        g = np.array([np.linalg.norm(self.p.get_grad(x_values[i]))
            for i in range(len(x_values))])
        c_e = np.array([np.linalg.norm(self.p.get_eq_const(x_values[i]))
            for i in range(len(x_values))])
        fig = plt.figure()
        plt.plot(np.arange(len(x_values)), g, label='Gradient Norm')
        #plt.plot(np.arange(len(x_values)), c_e, label='Equality Constraint Norm')
        plt.xticks(np.arange(len(x_values)))
        plt.xlabel('Iteration')
        plt.legend()
        plt.show()
        fig.savefig('./fig/al-pD-ag.eps', format='eps')
        print('\nProblem D-Eq, Augmented Lagrange')
        print('arg min v(x) =\n', x_values[-1])
        print('time =\n', end - start, 's')
    
    def test_lag_newton(self):
        x0 = np.array([[0], [1]])
        start = time.time()
        x_values = optimization.lagrange_newton(self.p, x0, tol=1e-6, hist=True)
        end = time.time()
        g = np.array([np.linalg.norm(self.p.get_grad(x_values[i]))
            for i in range(len(x_values))])
        c_e = np.array([np.linalg.norm(self.p.get_eq_const(x_values[i]))
            for i in range(len(x_values))])
        fig = plt.figure()
        plt.plot(np.arange(len(x_values)), g, label='Gradient Norm')
        #plt.plot(np.arange(len(x_values)), c_e, label='Equality Constraint Norm')
        plt.xticks(np.arange(len(x_values)))
        plt.xlabel('Iteration')
        plt.legend()
        #plt.show()
        fig.savefig('./fig/al-pD-ln.eps', format='eps')
        print('\nProblem D-Eq,Lagrange-Newton')
        print('arg min v(x) =\n', x_values[-1])
        print('time =\n', end - start, 's')


# class TestProblemDEq(unittest.TestCase):

#     def setUp(self):
#         v = lambda x: np.abs(x[0, 0] - 2) + np.abs(x[1, 0] - 2)
#         h1 = lambda x: x[0, 0] - x[1, 0]**2
#         h2 = lambda x: x[0, 0]**2 + x[1, 0]**2 - 1
#         self.p = optimization.Problem(v, eq_const=[h2, h1])
#         self.x_values = np.array([[np.sqrt(2)/2], [np.sqrt(2)/2]])

#     def test_aug_lag(self):
#         x0 = np.array([[0], [1]])
#         start = time.time()
#         x_values = optimization.augmented_lagrange(self.p, x0, tol=1e-4, tol_cost=1e-4, hist=True)
#         end = time.time()
#         g = np.array([np.linalg.norm(self.p.get_grad(x_values[i]))
#             for i in range(len(x_values))])
#         c_e = np.array([np.linalg.norm(self.p.get_eq_const(x_values[i]))
#             for i in range(len(x_values))])
#         fig = plt.figure()
#         plt.plot(np.arange(len(x_values)), g, label='Gradient Norm')
#         plt.plot(np.arange(len(x_values)), c_e, label='Equality Constraint Norm')
#         plt.xticks(np.arange(len(x_values)))
#         plt.xlabel('Iteration')
#         plt.legend()
#         #plt.show()
#         fig.savefig('./fig/al-pDEq.eps', format='eps')
#         print('\nProblem D-Eq, Augmented Lagrange')
#         print('arg min v(x) =\n', x_values[-1])
#         print('time =\n', end - start, 's')

class TestProblemE(unittest.TestCase):

    def setUp(self):
        v = lambda x: -x[0, 0] * x[1, 0]
        h1 = lambda x: -x[0, 0] - x[1, 0]**2 + 1
        h2 = lambda x: x[0, 0] + x[1, 0]
        self.p = optimization.Problem(v, ineq_const=[h1, h2])

    def test_penalty_function(self, tol=1e-4, tol_const=1e-4):
        x0 = np.array([[1], [1]])
        start = time.time()
        x_values = optimization.penalty_function(self.p, x0, hist=True)
        end = time.time()
        g = np.array([np.linalg.norm(self.p.get_grad(x_values[i]))
            for i in range(len(x_values))])
        c_i = np.array([np.linalg.norm(np.minimum(self.p.get_ineq_const(x_values[i]), 0))
            for i in range(len(x_values))])
        fig = plt.figure()
        plt.plot(np.arange(len(x_values)), g, label='Gradient Norm')
        #plt.plot(np.arange(len(x_values)), c_i, label='Inequality Constraint Norm')
        plt.xticks(np.arange(len(x_values)))
        plt.xlabel('Iteration')
        plt.legend()
        plt.show()
        fig.savefig('./fig/pe-pE.eps', format='eps')
        print('\nProblem E, Penalty Function')
        print('arg min v(x) =\n', x_values[-1])
        print('time =\n', end - start, 's')
    
    def test_lag_newton(self):
        x0 = np.array([[1], [1]])
        start = time.time()
        x_values = optimization.lagrange_newton(self.p, x0, tol=1e-6, hist=True)
        end = time.time()
        g = np.array([np.linalg.norm(self.p.get_grad(x_values[i]))
            for i in range(len(x_values))])
        c_e = np.array([np.linalg.norm(self.p.get_eq_const(x_values[i]))
            for i in range(len(x_values))])
        c_i = np.array([np.linalg.norm(np.minimum(self.p.get_ineq_const(x_values[i]), 0))
            for i in range(len(x_values))])
        fig = plt.figure()
        plt.plot(np.arange(len(x_values)), g, label='Gradient Norm')
        #plt.plot(np.arange(len(x_values)), c_e, label='Equality Constraint Norm')
        plt.xticks(np.arange(len(x_values)))
        plt.xlabel('Iteration')
        plt.legend()
        plt.show()
        fig.savefig('./fig/al-pE-ln.eps', format='eps')
        print('\nProblem E,Lagrange-Newton')
        print('arg min v(x) =\n', x_values[-1])
        print('time =\n', end - start, 's')
    
    def test_aug_lag(self):
        x0 = np.array([[1], [1]])
        start = time.time()
        x_values = optimization.augmented_lagrange(self.p, x0, tol=1e-4, tol_cost=1e-4, hist=True)
        end = time.time()
        g = np.array([np.linalg.norm(self.p.get_grad(x_values[i]))
            for i in range(len(x_values))])
        c_e = np.array([np.linalg.norm(self.p.get_eq_const(x_values[i]))
            for i in range(len(x_values))])
        fig = plt.figure()
        plt.plot(np.arange(len(x_values)), g, label='Gradient Norm')
        #plt.plot(np.arange(len(x_values)), c_e, label='Equality Constraint Norm')
        plt.xticks(np.arange(len(x_values)))
        plt.xlabel('Iteration')
        plt.legend()
        plt.show()
        fig.savefig('./fig/al-pE-ag.eps', format='eps')
        print('\nProblem E, Augmented Lagrange')
        print('arg min v(x) =\n', x_values[-1])
        print('time =\n', end - start, 's')


# class TestProblemEEq(unittest.TestCase):

#     def setUp(self):
#         v = lambda x: -x[0, 0] * x[1, 0]
#         #h1 = lambda x: -x[0, 0] - x[1, 0]**2 + 1
#         h1 = lambda x: -x[0, 0] - x[1, 0]**2 + 1
#         h2 = lambda x: x[0, 0] + x[1, 0]
#         self.p = optimization.Problem(v, eq_const=[h1, h2])

#     def test_aug_lag(self):
#         x0 = np.array([[1], [1]])
#         start = time.time()
#         x_values = optimization.augmented_lagrange(self.p, x0, tol=1e-4, tol_cost=1e-4, hist=True)
#         end = time.time()
#         g = np.array([np.linalg.norm(self.p.get_grad(x_values[i]))
#             for i in range(len(x_values))])
#         c_e = np.array([np.linalg.norm(np.minimum(self.p.get_eq_const(x_values[i]), 0))
#             for i in range(len(x_values))])
#         fig = plt.figure()
#         plt.plot(np.arange(len(x_values)), g, label='Gradient Norm')
#         plt.plot(np.arange(len(x_values)), c_e, label='Equality Constraint Norm')
#         plt.xticks(np.arange(len(x_values)))
#         plt.xlabel('Iteration')
#         plt.legend()
#         fig.savefig('./fig/al-pEEq.eps', format='eps')
#         print('\nProblem E-Eq, Augmented Lagrange')
#         print('arg min v(x) =\n', x_values[-1])
#         print('time =\n', end - start, 's')


#     def test_lag_new(self):
#         x0 = np.array([[0.7], [0.7]])
#         start = time.time()
#         x_values = optimization.lagrange_newton(self.p, x0, tol=1e-4, hist=True)
#         end = time.time()
#         g = np.array([np.linalg.norm(self.p.get_grad(x_values[i]))
#             for i in range(len(x_values))])
#         c_e = np.array([np.linalg.norm(np.minimum(self.p.get_eq_const(x_values[i]), 0))
#             for i in range(len(x_values))])
#         fig = plt.figure()
#         plt.plot(np.arange(len(x_values)), g, label='Gradient Norm')
#         plt.plot(np.arange(len(x_values)), c_e, label='Equality Constraint Norm')
#         plt.xticks(np.arange(len(x_values)))
#         plt.xlabel('Iteration')
#         plt.legend()
#         fig.savefig('./fig/ln-pEEq.eps', format='eps')
#         print('\nProblem E-Eq, Lagrange-Newton')
#         print('arg min v(x) =\n', x_values[-1])
#         print('time =\n', end - start, 's')

class TestProblemF(unittest.TestCase):

    def setUp(self):
        v = lambda x: np.log(x[0, 0]) - x[1, 0]
        h1 = lambda x: x[0, 0] - 1
        h2 = lambda x: x[0, 0]**2 + x[1, 0]**2 - 4
        self.p = optimization.Problem(v, eq_const=[h2], ineq_const=[h1])

    def test_penalty_function(self, tol=1e-4, tol_const=1e-4):
        x0 = np.array([[10], [10]])
        start = time.time()
        x_values = optimization.penalty_function(self.p, x0, hist=True)
        end = time.time()
        g = np.array([np.linalg.norm(self.p.get_grad(x_values[i]))
            for i in range(len(x_values))])
        c_i = np.array([np.linalg.norm(np.minimum(self.p.get_ineq_const(x_values[i]), 0))
            for i in range(len(x_values))])
        fig = plt.figure()
        plt.plot(np.arange(len(x_values)), g, label='Gradient Norm')
        #plt.plot(np.arange(len(x_values)), c_i, label='Inequality Constraint Norm')
        plt.xticks(np.arange(len(x_values)))
        plt.xlabel('Iteration')
        plt.legend()
        plt.show()
        fig.savefig('./fig/pe-pF.eps', format='eps')
        print('\nProblem E, Penalty Function')
        print('arg min v(x) =\n', x_values[-1])
        print('time =\n', end - start, 's')
    
    def test_lag_newton(self):
        x0 = np.array([[10], [10]])
        start = time.time()
        x_values = optimization.lagrange_newton(self.p, x0, tol=1e-6, hist=True)
        end = time.time()
        g = np.array([np.linalg.norm(self.p.get_grad(x_values[i]))
            for i in range(len(x_values))])
        c_e = np.array([np.linalg.norm(self.p.get_eq_const(x_values[i]))
            for i in range(len(x_values))])
        fig = plt.figure()
        plt.plot(np.arange(len(x_values)), g, label='Gradient Norm')
        #plt.plot(np.arange(len(x_values)), c_e, label='Equality Constraint Norm')
        plt.xticks(np.arange(len(x_values)))
        plt.xlabel('Iteration')
        plt.legend()
        plt.show()
        fig.savefig('./fig/al-pF-ln.eps', format='eps')
        print('\nProblem E,Lagrange-Newton')
        print('arg min v(x) =\n', x_values[-1])
        print('time =\n', end - start, 's')
    
    def test_aug_lag(self):
        x0 = np.array([[10], [10]])
        start = time.time()
        x_values = optimization.augmented_lagrange(self.p, x0, tol=1e-4, tol_cost=1e-4, hist=True)
        end = time.time()
        g = np.array([np.linalg.norm(self.p.get_grad(x_values[i]))
            for i in range(len(x_values))])
        c_e = np.array([np.linalg.norm(self.p.get_eq_const(x_values[i]))
            for i in range(len(x_values))])
        fig = plt.figure()
        plt.plot(np.arange(len(x_values)), g, label='Gradient Norm')
        #plt.plot(np.arange(len(x_values)), c_e, label='Equality Constraint Norm')
        plt.xticks(np.arange(len(x_values)))
        plt.xlabel('Iteration')
        plt.legend()
        plt.show()
        fig.savefig('./fig/al-pF-ag.eps', format='eps')
        print('\nProblem E, Augmented Lagrange')
        print('arg min v(x) =\n', x_values[-1])
        print('time =\n', end - start, 's')


#Call Class object with corresponding to the optimization problem A,B,C with TestProblem
TestProblemA().setUp()
TestProblemA.test_sec()
TestProblemA.test_sd()
TestProblemA.test_cg()

#Call Class object with corresponding to the optimization problem D,E,F with TestProblem
TestProblemE().setUp()
TestProblemE().setUp()
TestProblemE().test_aug_lag()
TestProblemE().test_lag_newton()
TestProblemE().test_penalty_function()


\end{lstlisting}